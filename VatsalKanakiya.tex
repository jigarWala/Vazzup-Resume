%%%%%%%%%%%%%%%%%%%%%%%%%%%%%%%%%%%%%%%%%
% Medium Length Professional CV
% LaTeX Template
% Version 2.0 (8/5/13)
%
% This template has been downloaded from:
% http://www.LaTeXTemplates.com
%
% Original author:
% Trey Hunner (http://www.treyhunner.com/)
%
% Important note:
% This template requires the resume.cls file to be in the same directory as the
% .tex file. The resume.cls file provides the resume style used for structuring the
% document.
%
%%%%%%%%%%%%%%%%%%%%%%%%%%%%%%%%%%%%%%%%%

%----------------------------------------------------------------------------------------
%	PACKAGES AND OTHER DOCUMENT CONFIGURATIONS
%----------------------------------------------------------------------------------------

\documentclass[a4paper]{resume} % Use the custom resume.cls style

\usepackage[left=0.75in,top=0.6in,right=0.75in,bottom=0.6in]{geometry} % Document margins

\name{Vatsal Kanakiya} % Your name
% \address{503, Sai Vaibhav \\ Ghatkopar(East) \\ Mumbai, MH - 400077} % Your address. Not required?
\address{Github~$\cdot$~{\em vazzup} \\ Codechef~$\cdot$~{\em vazzup} \\ LinkedIn~$\cdot$~{\em /in/vazzup}}
\address{(+91)~$\cdot$~98213~$\cdot$~72343 \\ vatsalkanakiya@gmail.com \\ vatsal.kanakiya@somaiya.edu} % Your phone number and email

\begin{document}

%----------------------------------------------------------------------------------------
%	EDUCATION SECTION
%----------------------------------------------------------------------------------------

\begin{rSection}{Education}

{\bf K. J. Somaiya College of Engineering, Mumbai} \hfill {\em August 2014 - Current} \\ 
B.Tech. in Computer Engineering \hfill{} \smallskip{}
\end{rSection}

%----------------------------------------------------------------------------------------
%	WORK EXPERIENCE SECTION
%----------------------------------------------------------------------------------------

\begin{rSection}{Experience}
\begin{rSubsection}{Morgan Stanley}{May 2017 - July 2017}{Summer Technology Intern}{Mumbai}
\item Worked on optimizing performance of internal tools written in C\#
\item Implemented a unique algorithm to store and retrieve hierarchichal data to and from a relational database.
\end{rSubsection}

%------------------------------------------------

\begin{rSubsection}{L.V. Prasad Eye Institute}{May 2016 - July 2016}{Summer Research Fellow}{Hyderabad}
\item Worked on software and hardware aspects of the Pediatric Perimeter ({\em Refer Projects}).
\item Learnt and implemented certain Image Processing and Machine Learning algorithms to achieve Gaze Tracking 
\end{rSubsection}

\end{rSection}
%----------------------------------------------------------------------------------------
%	EXTRACURRICULAR ACTIVITIES
%----------------------------------------------------------------------------------------
\begin{rSection}{Extracurricular Activities}
\begin{rProjects}{Committee Head, KJSCE Codecell}{July 2016 - May 2015}{Head of a team of 11 working to improve and promote participation in ACM ICPC and GSoC by conducting relevant workshops and seminars for fellow students}
\end{rProjects}
\end{rSection}

%----------------------------------------------------------------------------------------
% ASSORTED PROJECTS
%----------------------------------------------------------------------------------------
\begin{rSection}{Projects}
\begin{rProjects}{Automated Gait Generation for Simulated Machines}{July 2017 - Current}{L.Y. Project to generate gait for simulated machines using Self Modelling and Neural Networks}
\end{rProjects}
\begin{rProjects}{Pediatric Perimeter}{May - July 2016}{A first of its kind device to measure the visual fields of infants, detect defects \& provide preventive medical help}
\end{rProjects}
\begin{rProjects}{Controllers Workbench}{May - July 2017}{A desktop application to visualize, modify and analyse financial data which can handle hierarchical data of the magnitude of 10\textsuperscript{8} rows and 400 columns.}
\end{rProjects}
\end{rSection}

%----------------------------------------------------------------------------------------
%	COMPETITIONS
%----------------------------------------------------------------------------------------
\begin{rSection}{Competitions}
%----------------------------------------------------------------------------------------
\begin{rCompetitions}{DJSCE Code Uncode}{March 2016}{Mumbai}{3\textsuperscript{rd}/170 participants}
\end{rCompetitions}
%----------------------------------------------------------------------------------------
\begin{rCompetitions}{ACM ICPC Regionals}{December 2016}{Chennai}{73\textsuperscript{rd}/867 teams}
\end{rCompetitions}
%---------------------------------------------------
\begin{rCompetitions}{ACM ICPC Regionals}{December 2016}{Amritapuri}{219\textsuperscript{th}/1981 teams}
\end{rCompetitions}
%---------------------------------------------------
\begin{rCompetitions}{ACM ICPC Regionals}{December 2015}{Amritapuri}{228\textsuperscript{th}/1572 teams}
\end{rCompetitions}
%---------------------------------------------------
\end{rSection}

%----------------------------------------------------------------------------------------
%	TECHNICAL STRENGTHS SECTION
%----------------------------------------------------------------------------------------

\begin{rSection}{Technical Strengths}

\begin{tabular}{ @{} >{\bfseries}l @{\hspace{6ex}} l }
Computer Languages & C/C++, Java, Python3, C\#, Javascript \\
Protocols \& APIs & XML, JSON, SOAP, REST \\
Databases & MySQL, PostgreSQL, Microsoft SQL, Redis \\
Frameworks & Angular, Django \\
AI Frameworks & Tensorflow, Keras \\ 
Tools & Linux, Git, Vim, Make, CMake, IntelliJ Idea, Android Studio {\em etc.}\\
\end{tabular}

\end{rSection}

%----------------------------------------------------------------------------------------
%	EXAMPLE SECTION
%----------------------------------------------------------------------------------------

%\begin{rSection}{Section Name}

%Section content\ldots

%\end{rSection}

%----------------------------------------------------------------------------------------

\end{document}
